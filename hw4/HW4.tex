%%%
%%%
\documentclass[12pt]{extarticle}
\usepackage{fullpage,latexsym,picinpar,amsmath,amsfonts, amsthm}
\input{macros.tex}

%\pagestyle{empty}

\usepackage[hidelinks]{hyperref} 
\usepackage{makeidx}
\usepackage{enumerate}
\usepackage{stmaryrd}
\usepackage{mathrsfs}
\usepackage{mdframed}
\usepackage{lipsum}
\usepackage{anyfontsize}
\everymath{\displaystyle}
%
\textwidth 6in
\oddsidemargin 0.25in
\evensidemargin 0.25in 
%
\title{Math 120 Optimization: Homework 4}
\author{Stanley Cohen (scohe001)}
%\address{Department of Mathematics, Vanderbilt University, 1326 Stevenson Center, Nashville, TN 37240}
%\email{chenxu.wen@vanderbilt.edu}
\date{}
%\thanks{}
%\subjclass{}
%\renewcommand{\subjclassname}{\textup{2000} Mathematics Subject Classification}
%\keywords{}
%\dedicatory{}

%\makeindex

%
\newtheorem{thm}{Theorem}
\newtheorem*{thm*}{Theorem}
\newtheorem*{cor*}{Corollary}
\newtheorem{prop}[thm]{Proposition}
\newtheorem{cor}[thm]{Corollary}
\newtheorem{lem}[thm]{Lemma}
\theoremstyle{definition}
\newtheorem{defn}[thm]{Definition}
\newtheorem{rem}[thm]{Remark}
\newtheorem{examp}[thm]{Example}
\newtheorem{claim}[thm]{Claim}
\newtheorem{exer}[thm]{Exercise}
\newtheorem{prob}[thm]{Problem}
\newtheorem{open}[thm]{Open Problem}
\newtheorem{note}[thm]{Notation}
\newtheorem{que}[thm]{Question}
\def\R{{\mathbb{R}}}


%
\newcommand{\actson}{{\curvearrowright}}



%
\begin{document}



\maketitle

\begin{problem} Consider the system of equations 

	\[\begin{bmatrix}
	2&-1&2&-1&3\\1&2&3&1&0\\1&0&-2&0&-5
	\end{bmatrix}\begin{bmatrix}
	x_1\\x_2\\x_3\\x_4\\x_5
	\end{bmatrix}=\begin{bmatrix}
	14\\5\\-10
	\end{bmatrix}.\]
	Find one basic solution. State explicitly which basis you are using.

\end{problem}

\begin{problem} Consider the system of equations 

	\[\begin{bmatrix}
	2&1&-1&0\\1&3&1&3
	\end{bmatrix}\begin{bmatrix}
	x_1\\x_2\\x_3\\x_4
	\end{bmatrix}=\begin{bmatrix}
	3\\5
	\end{bmatrix}.\]
	Find all basic solutions. 

\end{problem}

\begin{problem} Solve the linear programming problem:

	\begin{align*}
	\text{minimize } \quad &x_1-x_3\\
	\text{subject to} \quad & \begin{bmatrix}
	2&1&-1&0\\1&3&1&3
	\end{bmatrix}\begin{bmatrix}
	x_1\\x_2\\x_3\\x_4
	\end{bmatrix}=\begin{bmatrix}
	3\\5
	\end{bmatrix}\\
	& \vec{x}\geq \vec{0}.
	\end{align*}
	You may use the result from the previous problem.

\end{problem}

\begin{problem} Prove that any subspace $V$ of $\R^n$ is a convex set.

\end{problem}

\begin{problem} Consider the system of equations 

	\[\begin{bmatrix}
	2&-1&2&-1&3\\1&2&3&1&0\\1&0&-2&0&-5
	\end{bmatrix}\begin{bmatrix}
	x_1\\x_2\\x_3\\x_4\\x_5
	\end{bmatrix}=\begin{bmatrix}
	14\\5\\-10
	\end{bmatrix}.\]
	Write down the augmented matrix of the system and convert it to the canonical form using elementary row operations. State explicitly which basis your canonical form is with respect to.

\end{problem}

\begin{problem} Consider the system of equations 

	\[\begin{bmatrix}
	1&1&1&0&1\\
	0&2&0&0&2\\
	3&4&0&1&1\\
	0&0&0&1&-1
	\end{bmatrix}\begin{bmatrix}
	x_1\\x_2\\x_3\\x_4\\x_5
	\end{bmatrix}=\begin{bmatrix}
	0\\2\\0\\0
	\end{bmatrix}.\]
	Write down the augmented matrix of the system and convert it to the canonical form using elementary row operations(Hint: the first 4 columns are linearly independent). 

\end{problem}


\small
\bibliographystyle{amsalpha}
\bibliography{ref}


\end{document}





