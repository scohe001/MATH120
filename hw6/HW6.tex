%%%
%%%
\documentclass[12pt]{extarticle}
\usepackage{fullpage,latexsym,picinpar,amsmath,amsfonts, amsthm}
\input{macros.tex}

%\pagestyle{empty}

\usepackage[hidelinks]{hyperref} 
\usepackage{makeidx}
\usepackage{enumerate}
\usepackage{stmaryrd}
\usepackage{mathrsfs}
\usepackage{mdframed}
\usepackage{lipsum}
\usepackage{anyfontsize}
\everymath{\displaystyle}
%
\textwidth 6in
\oddsidemargin 0.25in
\evensidemargin 0.25in 
%
\title{Math 120 Optimization: Homework 6}
\author{Stanley Cohen (scohe001)}
%\address{Department of Mathematics, Vanderbilt University, 1326 Stevenson Center, Nashville, TN 37240}
%\email{chenxu.wen@vanderbilt.edu}
\date{}
%\thanks{}
%\subjclass{}
%\renewcommand{\subjclassname}{\textup{2000} Mathematics Subject Classification}
%\keywords{}
%\dedicatory{}

%\makeindex

%
\newtheorem{thm}{Theorem}
\newtheorem*{thm*}{Theorem}
\newtheorem*{cor*}{Corollary}
\newtheorem{prop}[thm]{Proposition}
\newtheorem{cor}[thm]{Corollary}
\newtheorem{lem}[thm]{Lemma}
\theoremstyle{definition}
\newtheorem{defn}[thm]{Definition}
\newtheorem{rem}[thm]{Remark}
\newtheorem{examp}[thm]{Example}
\newtheorem{claim}[thm]{Claim}
\newtheorem{exer}[thm]{Exercise}
\newtheorem{prob}[thm]{Problem}
\newtheorem{open}[thm]{Open Problem}
\newtheorem{note}[thm]{Notation}
\newtheorem{que}[thm]{Question}
\def\R{{\mathbb{R}}}


%
\newcommand{\actson}{{\curvearrowright}}

\makeatletter
\renewcommand*\env@matrix[1][*\c@MaxMatrixCols c]{%
  \hskip -\arraycolsep
  \let\@ifnextchar\new@ifnextchar
  \array{#1}}
\makeatother


%
\begin{document}



\maketitle

\begin{problem} Consider the following linear programming problem

  	\begin{align*}
	\text{minimize } &-x_1+2x_2-x_3\\
	\text{subject to } & x_1+3x_2+x_4=4\\
	&2x_1+6x_2+x_3+x_4=5\\
	&x_1,x_2,x_3,x_4\geq 0.
	\end{align*}

	\begin{description}
		\item{(a)} Form the associated artificial problem and carry out the Phase I in the Two-Phase Simplex Method.\\

		The artificial problem will look like:

		\begin{align*}
		\text{minimize } &y_1 + y_2\\
		\text{subject to } & x_1+3x_2+x_4+y_1=4\\
		&2x_1+6x_2+x_3+x_4+y_2=5\\
		&x_1,x_2,x_3,x_4,y_1,y_2\geq 0.
		\end{align*}

		Meaning the tableau will look like:

		$$\begin{bmatrix}[cccccc|c] 1&3&0&1&1&0&4\\ 2&6&1&1&0&1&5 \\ 0&0&0&0&1&1&0 \end{bmatrix}$$

		First we'll remove the positive 1's in the bottom row to get:

		$$\begin{bmatrix}[cccccc|c] 1&3&0&1&1&0&4\\ 2&6&1&1&0&1&5 \\ -3&-9&-1&-2&0&0&-9 \end{bmatrix}$$

		\item{(b)} From the final tableau for Phase I, find the initial canonical tableau for phase II (you don't need to solve the original problem).\\

		After running Simplex, we'll get the tableau:

		$$\begin{bmatrix}[cccccc|c] 0&0&-1&1&2&-1&3\\ 1/3&1&1/3&0&-1/3&1/3&1/3\\ 0&0&0&0&1&1&0 \end{bmatrix}$$

		To transform this back into the original problem, we can reconstruct the tableau using this solution:

		$$\begin{bmatrix}[cccc|c] 0&0&-1&1&3\\ 1/3&1&1/3&0&1/3\\ -1&2&-1&0&0 \end{bmatrix}$$

	\end{description}

\end{problem}

\begin{problem} Consider the following linear programming problem

  	\begin{align*}
	\text{minimize } &-2x-3y-4z\\
	\text{subject to } & 3x+2y+z=10\\
	&2x+5y+3z=15\\
	&x,y,z\geq 0.
	\end{align*}

	\begin{description}
		\item{(a)} Form the associated artificial problem and carry out the Phase I in the Two-Phase Simplex Method.\\

		The artificial problem will look like:

		  	\begin{align*}
			\text{minimize } &y_1+y_2\\
			\text{subject to } & 3x+2y+z+y_1=10\\
			&2x+5y+3z+y_2=15\\
			&x,y,z,y_1,y_2\geq 0.
			\end{align*}

		Meaning the tableau will look like:

		$$\begin{bmatrix}[ccccc|c] 3&2&1&1&0&10\\ 2&5&3&0&1&15\\ 0&0&0&1&1&0 \end{bmatrix}$$

		\item{(b)} From the final tableau for Phase I, find the initial canonical tableau for phase II (you don't need to solve the original problem).

		Runnin Simplex on this will net us a matrix looking like:
		
		$$\begin{bmatrix}[ccccc|c] 1&0&-1/11&1&-2/11&20/11\\ 0&1&7/11&-2/5&3/11&25/11\\ 0&0&0&1&1&0 \end{bmatrix}$$

		Going back to the original problem for phase 2 gives us:

		$$\begin{bmatrix}[ccc|c] 1&0&-1/11&20/11\\ 0&1&7/11&25/11\\ -2&-3&-4&0 \end{bmatrix}$$

	\end{description}

\end{problem}

\begin{problem} Consider a standard form linear programming problem with 

	\[A=\begin{bmatrix}
	0&2&0&1\\1&1&0&0\\0&3&1&0
	\end{bmatrix},\vec{b}=\begin{bmatrix}
	7\\8\\9
	\end{bmatrix},\vec{c}=\begin{bmatrix}
	6\\c_2\\4\\5
	\end{bmatrix}.\]

	Suppose that we are told that the reduced cost coefficient vector corresponding to some basis is $\vec{r}^T=[0,1,0,0]$.\\

	\begin{description}
		\item{(a)} Find an optimal feasible solution to the problem;\\

		$(x_1,x_2,x_3,x_4) = (8, 0, 9, 7)$ with a value of $48+36+35 = 119$

		\item{(b)} Find $c_2$.

		$c_2 = 6 + 12 + 10 + 1 = 29$

	\end{description}

\end{problem}

\begin{problem} Consider the linear program

 \begin{align*}
\text{minimize } &4x_1+3x_2\\
\text{subject to } & 5x_1+x_2\geq 11\\
&2x_1+x_2\geq 8\\
&x_1+2x_2\geq 7\\
&x_1,x_2\geq 0.
\end{align*}

Write down the corresponding dual problem.

For this problem, 

	$$A=\begin{bmatrix}
	5&1\\2&1\\1&2
	\end{bmatrix},\vec{b}=\begin{bmatrix}
	11\\8\\7
	\end{bmatrix},\vec{c}=\begin{bmatrix}
	4\\3
	\end{bmatrix}.$$

So the dual will be:
	\begin{align*}
	\text{max } &\vec{b} \cdot \vec{y}\\
	\text{subject to } & A^T \vec{y} \leq \vec{c}\\
	&\vec{y}\geq 0.
	\end{align*}

Where

	$$A^T=\begin{bmatrix}
	5&2&1\\1&1&2
	\end{bmatrix},\vec{c}=\begin{bmatrix}
	4\\3
	\end{bmatrix},\vec{b}=\begin{bmatrix}
	11\\8\\7
	\end{bmatrix}.$$

\end{problem}


\small
\bibliographystyle{amsalpha}
\bibliography{ref}


\end{document}





