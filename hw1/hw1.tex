\documentclass{article}

\usepackage{fullpage,latexsym,picinpar,amsmath,amsfonts}

\input{macros.tex}

\begin{document}
 
 
\centerline{\large \bf MATH120 Homework 1}
\centerline{Stanley Cohen (scohe001)}
 
\begin{problem}
    Find $det(A)$ given the following: $A=
    \begin{bmatrix}
        1 & 0 & -1\\
        2 & 4 & 0\\
        0 & 0 & 7
    \end{bmatrix}$\\

    Taking the determinent in respect to the bottom row will yield: 
    $$(0)det(\begin{bmatrix} 0 & -1 \\ 4 & 0 \end{bmatrix}) - 
    (0)det(\begin{bmatrix} 1 & -1 \\ 2 & 0 \end{bmatrix}) + 
    (7)det(\begin{bmatrix} 1 & 0 \\ 2 & 4 \end{bmatrix}) = $$

    $$(0) - (0) + (7)(4) = 28$$
\end{problem}
    
\begin{problem}
    Let $A=
    \begin{bmatrix}
        1 & 2 & -1\\
        3 & 4 & 2
    \end{bmatrix},
    \quad
    B= \begin{bmatrix}
        1\\
        1\\
        -1
    \end{bmatrix}$
    
    \begin{description}
    \item{(a)} Find $A^T$ and $B^T$.\\

        $A^T = \begin{bmatrix} 1 & 3 \\ 2 & 4 \\ -1 & 2 \end{bmatrix}, \quad
        B^T = \begin{bmatrix} 1 & 1 & -1 \end{bmatrix}$

    \item{(b)} Find $AB$ and $BA$ (if the product is not defined, write \textit{NOT DEFINED})\\

        $AB = \begin{bmatrix} (1, 2, -1) \cdot (1, 1, -1) \\ (3, 4, 2) \cdot (1, 1, -1) \end{bmatrix} = 
            \begin{bmatrix} (1)(1) + (2)(1) + (-1)(-1) \\ (3)(1) + (4)(1) + (2)(-1) \end{bmatrix} = 
            \begin{bmatrix} 4 \\ 5 \end{bmatrix}$\\

        $BA$ is not defined as you are attempting to multiply a 3x1 matrix by a 2x3. 
        The columns of $B$ (1) are not equal to the rows of $A$ (2).
    \end{description}

\end{problem}

\begin{problem}
    Let $L: {\rm I\!R}^2 \mapsto {\rm I\!R}^3$ be a linear transormation subject to:
    $$
    L(\begin{bmatrix} 1\\ 0 \end{bmatrix}) = 
        \begin{bmatrix} 1\\ 2\\ 1 \end{bmatrix}
    \quad\quad
    L(\begin{bmatrix} 0\\ 1 \end{bmatrix}) = 
        \begin{bmatrix} -1\\ 0\\ 0 \end{bmatrix}
    $$
    
    \begin{description}

    \item{(a)} Find the matrix representation of $L$ with respect to the standard bases
                    of ${\rm I\!R}^2$ and ${\rm I\!R}^3$.\\

        We know what the standard basis of ${\rm I\!R}^2$ maps to under L, so finding the 
        matrix representation for the tranformation becomes trivial: $L(\vec{x}) = A\vec{x}$
        where $A=
        \begin{bmatrix} 1 & -1 \\ 2 & 0 \\ 1 & 0 \end{bmatrix}$
    
    \item{(b)} Find the matrix representation of $L$ with respect to the basis 
                $\left \{ \vec{e}_1, \vec{e}_2 \right \}$ of ${\rm I\!R}^2$ and the standard basis
                of ${\rm I\!R}^3$, where $\vec{e}_1 = \begin{bmatrix}1\\ 1 \end{bmatrix}$, 
                $\vec{e}_2 = \begin{bmatrix}0\\ -1 \end{bmatrix}$.\\

        We can say $\vec{x} = C[\vec{x}]_B$ where $B$ is the basis $\left \{ \vec{e}_1, \vec{e}_2 \right \}$ 
        and $C$ is the matrix formed by using the basis vectors as column vectors, $C = \begin{bmatrix} 1 & 0 \\ 1 & -1 \end{bmatrix}$.\\
        The transformation we're looking for here maps a vector with respect to the basis $B$ to a vector with respect to the standard basis of ${\rm I\!R}^3$. We can call this transformation $D$.\\
        Since both $D$ and $L$ map to the same vectors, we know for any $\vec{x}$, $D([\vec{x}]_B)=L(\vec{x})=A\vec{x}=AC[\vec{x}]_B$.\\

        $AC= \begin{bmatrix} 1 & -1 \\ 2 & 0 \\ 1 & 0 \end{bmatrix} \begin{bmatrix} 1 & 0 \\ 1 & -1 \end{bmatrix} = 
        \begin{bmatrix} (1, -1)\cdot(1, 1) & (1, -1)\cdot(0, -1) \\
                        (2, 0)\cdot(1, 1) & (2, 0)\cdot(0, -1) \\
                        (1, 0)\cdot(1, 1) & (1, 0)\cdot(0, -1) \end{bmatrix} =
        \begin{bmatrix} 0 & 1 \\ 2 & 0 \\ 1 & 0 \end{bmatrix}$

    \end{description}

\end{problem}

\end{document}