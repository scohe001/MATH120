\documentclass{article}

\usepackage{fullpage,latexsym,picinpar,amsmath,amsfonts}

           

%%%%%%%%%%%%%%%%%%%%%%%%%%%%%%%%%%%%%%%%%%%%%%%%%%%%%%%%%%%%%%%%%%%%%%%%%%%%%%%%%%%
%%%%%%%%%%%  LETTERS 
%%%%%%%%%%%%%%%%%%%%%%%%%%%%%%%%%%%%%%%%%%%%%%%%%%%%%%%%%%%%%%%%%%%%%%%%%%%%%%%%%%%

\newcommand{\barx}{{\bar x}}
\newcommand{\bary}{{\bar y}}
\newcommand{\barz}{{\bar z}}
\newcommand{\bart}{{\bar t}}

\newcommand{\bfP}{{\bf{P}}}

%%%%%%%%%%%%%%%%%%%%%%%%%%%%%%%%%%%%%%%%%%%%%%%%%%%%%%%%%%%%%%%%%%%%%%%%%%%%%%%%%%%
%%%%%%%%%%%%%%%%%%%%%%%%%%%%%%%%%%%%%%%%%%%%%%%%%%%%%%%%%%%%%%%%%%%%%%%%%%%%%%%%%%%
                                                                                
\newcommand{\parend}[1]{{\left( #1  \right) }}
\newcommand{\spparend}[1]{{\left(\, #1  \,\right) }}
\newcommand{\angled}[1]{{\left\langle #1  \right\rangle }}
\newcommand{\brackd}[1]{{\left[ #1  \right] }}
\newcommand{\spbrackd}[1]{{\left[\, #1  \,\right] }}
\newcommand{\braced}[1]{{\left\{ #1  \right\} }}
\newcommand{\leftbraced}[1]{{\left\{ #1  \right. }}
\newcommand{\floor}[1]{{\left\lfloor #1\right\rfloor}}
\newcommand{\ceiling}[1]{{\left\lceil #1\right\rceil}}
\newcommand{\barred}[1]{{\left|#1\right|}}
\newcommand{\doublebarred}[1]{{\left|\left|#1\right|\right|}}
\newcommand{\spaced}[1]{{\, #1\, }}
\newcommand{\suchthat}{{\spaced{|}}}
\newcommand{\numof}{{\sharp}}
\newcommand{\assign}{{\,\leftarrow\,}}
\newcommand{\myaccept}{{\mbox{\tiny accept}}}
\newcommand{\myreject}{{\mbox{\tiny reject}}}
\newcommand{\blanksymbol}{{\sqcup}}
                                                                                                                         
\newcommand{\veps}{{\varepsilon}}
\newcommand{\Sigmastar}{{\Sigma^\ast}}
                           
\newcommand{\half}{\mbox{$\frac{1}{2}$}}    
\newcommand{\threehalfs}{\mbox{$\frac{3}{2}$}}   
\newcommand{\domino}[2]{\left[\frac{#1}{#2}\right]}  

%%%%%%%%%%%% complexity classes

\newcommand{\PP}{\mathbb{P}}
\newcommand{\NP}{\mathbb{NP}}
\newcommand{\PSPACE}{\mathbb{PSPACE}}
\newcommand{\coNP}{\textrm{co}\mathbb{NP}}
\newcommand{\DLOG}{\mathbb{L}}
\newcommand{\NLOG}{\mathbb{NL}}
\newcommand{\NL}{\mathbb{NL}}

%%%%%%%%%%% decision problems

\newcommand{\PCP}{\sc{PCP}}
\newcommand{\Path}{\sc{Path}}
\newcommand{\GenGeo}{\sc{Generalized Geography}}

\newcommand{\malytm}{{\mbox{\tiny TM}}}
\newcommand{\malycfg}{{\mbox{\tiny CFG}}}
\newcommand{\Atm}{\mbox{\rm A}_\malytm}
\newcommand{\complAtm}{{\overline{\mbox{\rm A}}}_\malytm}
\newcommand{\AllCFG}{{\mbox{\sc All}}_\malycfg}
\newcommand{\complAllCFG}{{\overline{\mbox{\sc All}}}_\malycfg}
\newcommand{\complL}{{\bar L}}
\newcommand{\TQBF}{\mbox{\sc TQBF}}
\newcommand{\SAT}{\mbox{\sc SAT}}

%%%%%%%%%%%%%%%%%%%%%%%%%%%%%%%%%%%%%%%%%%%%%%%%%%%%%%%%%%%%%%%%%%%%%%%%%%%%%%%%%%%
%%%%%%%%%%%%%%% for homeworks
%%%%%%%%%%%%%%%%%%%%%%%%%%%%%%%%%%%%%%%%%%%%%%%%%%%%%%%%%%%%%%%%%%%%%%%%%%%%%%%%%%%

\newcommand{\student}[2]{%
{\noindent\Large{ \emph{#1} SID {#2} } \hfill} \vskip 0.1in}

\newcommand{\assignment}[1]{\medskip\centerline{\large\bf CS 111 ASSIGNMENT {#1}}}

\newcommand{\duedate}[1]{{\centerline{due {#1}\medskip}}}     

\newcounter{problemnumber}                                                                                 

\newenvironment{problem}{{\vskip 0.1in \noindent
              \bf Problem~\addtocounter{problemnumber}{1}\arabic{problemnumber}:}}{}

\newcounter{solutionnumber}

\newenvironment{solution}{{\vskip 0.1in \noindent
             \bf Solution~\addtocounter{solutionnumber}{1}\arabic{solutionnumber}:}}
				{\ \newline\smallskip\lineacross\smallskip}

\newcommand{\lineacross}{\noindent\mbox{}\hrulefill\mbox{}}

\newcommand{\decproblem}[3]{%
\medskip
\noindent
\begin{list}{\hfill}{\setlength{\labelsep}{0in}
                       \setlength{\topsep}{0in}
                       \setlength{\partopsep}{0in}
                       \setlength{\leftmargin}{0in}
                       \setlength{\listparindent}{0in}
                       \setlength{\labelwidth}{0.5in}
                       \setlength{\itemindent}{0in}
                       \setlength{\itemsep}{0in}
                     }
\item{{{\sc{#1}}:}}
                \begin{list}{\hfill}{\setlength{\labelsep}{0.1in}
                       \setlength{\topsep}{0in}
                       \setlength{\partopsep}{0in}
                       \setlength{\leftmargin}{0.5in}
                       \setlength{\labelwidth}{0.5in}
                       \setlength{\listparindent}{0in}
                       \setlength{\itemindent}{0in}
                       \setlength{\itemsep}{0in}
                       }
                \item{{\em Instance:\ }}{#2}
                \item{{\em Query:\ }}{#3}
                \end{list}
\end{list}
\medskip
}

%%%%%%%%%%%%%%%%%%%%%%%%%%%%%%%%%%%%%%%%%%%%%%%%%%%%%%%%%%%%%%%%%%%%%%%%%%%%%%%%%%%
%%%%%%%%%%%%% for quizzes
%%%%%%%%%%%%%%%%%%%%%%%%%%%%%%%%%%%%%%%%%%%%%%%%%%%%%%%%%%%%%%%%%%%%%%%%%%%%%%%%%%%

\newcommand{\quizheader}{ {\large NAME: \hskip 3in SID:\hfill}
                                \newline\lineacross \medskip }


%%%%%%%%%%%%%%%%%%%%%%%%%%%%%%%%%%%%%%%%%%%%%%%%%%%%%%%%%%%%%%%%%%%%%%%%%%%%%%%%%%%
%%%%%%%%%%%%% for final
%%%%%%%%%%%%%%%%%%%%%%%%%%%%%%%%%%%%%%%%%%%%%%%%%%%%%%%%%%%%%%%%%%%%%%%%%%%%%%%%%%%

\newcommand{\namespace}{\noindent{\Large NAME: \hfill SID:\hskip 1.5in\ }\\\medskip\noindent\mbox{}\hrulefill\mbox{}}



\begin{document}
 
 
\centerline{\large \bf MATH120 Homework 1}
\centerline{Stanley Cohen (scohe001)}
 
\begin{problem}
    Find $det(A)$ given the following: $A=
    \begin{bmatrix}
        1 & 0 & -1\\
        2 & 4 & 0\\
        0 & 0 & 7
    \end{bmatrix}$\\

    Taking the determinent in respect to the bottom row will yield: 
    $$(0)det(\begin{bmatrix} 0 & -1 \\ 4 & 0 \end{bmatrix}) - 
    (0)det(\begin{bmatrix} 1 & -1 \\ 2 & 0 \end{bmatrix}) + 
    (7)det(\begin{bmatrix} 1 & 0 \\ 2 & 4 \end{bmatrix}) = $$

    $$(0) - (0) + (7)(4) = 28$$
\end{problem}
    
\begin{problem}
    Let $A=
    \begin{bmatrix}
        1 & 2 & -1\\
        3 & 4 & 2
    \end{bmatrix},
    \quad
    B= \begin{bmatrix}
        1\\
        1\\
        -1
    \end{bmatrix}$
    
    \begin{description}
    \item{(a)} Find $A^T$ and $B^T$.\\

        $A^T = \begin{bmatrix} 1 & 3 \\ 2 & 4 \\ -1 & 2 \end{bmatrix}, \quad
        B^T = \begin{bmatrix} 1 & 1 & -1 \end{bmatrix}$

    \item{(b)} Find $AB$ and $BA$ (if the product is not defined, write \textit{NOT DEFINED})\\

        $AB = \begin{bmatrix} (1, 2, -1) \cdot (1, 1, -1) \\ (3, 4, 2) \cdot (1, 1, -1) \end{bmatrix} = 
            \begin{bmatrix} (1)(1) + (2)(1) + (-1)(-1) \\ (3)(1) + (4)(1) + (2)(-1) \end{bmatrix} = 
            \begin{bmatrix} 4 \\ 5 \end{bmatrix}$\\

        $BA$ is not defined as you are attempting to multiply a 3x1 matrix by a 2x3. 
        The columns of $B$ (1) are not equal to the rows of $A$ (2).
    \end{description}

\end{problem}

\begin{problem}
    Let $L: {\rm I\!R}^2 \mapsto {\rm I\!R}^3$ be a linear transormation subject to:
    $$
    L(\begin{bmatrix} 1\\ 0 \end{bmatrix}) = 
        \begin{bmatrix} 1\\ 2\\ 1 \end{bmatrix}
    \quad\quad
    L(\begin{bmatrix} 0\\ 1 \end{bmatrix}) = 
        \begin{bmatrix} -1\\ 0\\ 0 \end{bmatrix}
    $$
    
    \begin{description}

    \item{(a)} Find the matrix representation of $L$ with respect to the standard bases
                    of ${\rm I\!R}^2$ and ${\rm I\!R}^3$.\\

        We know what the standard basis of ${\rm I\!R}^2$ maps to under L, so finding the 
        matrix representation for the tranformation becomes trivial: $L(\vec{x}) = A\vec{x}$
        where $A=
        \begin{bmatrix} 1 & -1 \\ 2 & 0 \\ 1 & 0 \end{bmatrix}$
    
    \item{(b)} Find the matrix representation of $L$ with respect to the basis 
                $\left \{ \vec{e}_1, \vec{e}_2 \right \}$ of ${\rm I\!R}^2$ and the standard basis
                of ${\rm I\!R}^3$, where $\vec{e}_1 = \begin{bmatrix}1\\ 1 \end{bmatrix}$, 
                $\vec{e}_2 = \begin{bmatrix}0\\ -1 \end{bmatrix}$.\\

        We can say $\vec{x} = C[\vec{x}]_B$ where $B$ is the basis $\left \{ \vec{e}_1, \vec{e}_2 \right \}$ 
        and $C$ is the matrix formed by using the basis vectors as column vectors, $C = \begin{bmatrix} 1 & 0 \\ 1 & -1 \end{bmatrix}$.\\
        The transformation we're looking for here maps a vector with respect to the basis $B$ to a vector with respect to the standard basis of ${\rm I\!R}^3$. We can call this transformation $D$.\\
        Since both $D$ and $L$ map to the same vectors, we know for any $\vec{x}$, $D([\vec{x}]_B)=L(\vec{x})=A\vec{x}=AC[\vec{x}]_B$.\\

        $AC= \begin{bmatrix} 1 & -1 \\ 2 & 0 \\ 1 & 0 \end{bmatrix} \begin{bmatrix} 1 & 0 \\ 1 & -1 \end{bmatrix} = 
        \begin{bmatrix} (1, -1)\cdot(1, 1) & (1, -1)\cdot(0, -1) \\
                        (2, 0)\cdot(1, 1) & (2, 0)\cdot(0, -1) \\
                        (1, 0)\cdot(1, 1) & (1, 0)\cdot(0, -1) \end{bmatrix} =
        \begin{bmatrix} 0 & 1 \\ 2 & 0 \\ 1 & 0 \end{bmatrix}$

    \end{description}

\end{problem}

\end{document}